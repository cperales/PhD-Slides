\pdfminorversion=4

\documentclass{beamer}
\usepackage[utf8]{inputenc}
\usepackage[spanish]{babel}
%\usepackage[english]{babel}

\usepackage{beamerthemeULoyola}
\usepackage{graphicx}
\usepackage{booktabs}
\usepackage{blindtext}

%% PRESENTATION CONFIGURATION PARAMETERS %%%%%%%%%%%%%%%%%%%%%%%%%%%%%%%%%%%%%%%
\titlebackgroundfile{images/plantilla_portada}
\framebackgroundfile{images/plantilla_slide}
\definecolor{azulloyola}{HTML}{023E83}
\definecolor{azulloyolaclaro}{HTML}{9497B9}
\definecolor{gris}{HTML}{4C4C4C}
\definecolor{grisclaro}{HTML}{EBEBEB}
\definecolor{examplefrente}{HTML}{218E58}
\definecolor{examplefondo}{HTML}{C2D8CD}
\definecolor{alertfrente}{HTML}{FF0000}
\definecolor{alertfondo}{HTML}{E7BDBD}


\usefonttheme{structurebold}
\setbeamercolor{author in head/foot}{fg=white}
\setbeamercolor{title in head/foot}{fg=white}
\setbeamercolor{section in head/foot}{fg=azulloyola}
\setbeamercolor{normal text}{fg=gris}
\setbeamercolor{frametitle}{fg=azulloyola}
% \setbeamerfont{block title}{size={}}
\setbeamerfont{author}{size=\footnotesize}
\setbeamerfont{date}{size=\footnotesize}
\setbeamertemplate{itemize item}[circle]
\setbeamertemplate{itemize subitem}[triangle]
\setbeamertemplate{itemize subsubitem}[square]
\setbeamertemplate{itemize subsubsubitem}[ball]
\setbeamercolor{itemize item}{fg=azulloyola}
\setbeamercolor{itemize subitem}{fg=azulloyola}
\setbeamercolor{itemize subsubitem}{fg=azulloyola}
\setbeamercolor{itemize subsubsubitem}{fg=azulloyola}
\setbeamercolor{enumerate item}{fg=azulloyola}
\setbeamercolor{enumerate subitem}{fg=azulloyola}
\setbeamercolor{enumerate subsubitem}{fg=azulloyola}
\setbeamercolor{enumerate subsubsubitem}{fg=azulloyola}
% \setbeamercolor{alerted text}{fg=azulloyola}
% \setbeamerfont{alerted text}{series=\bfseries}

\setbeamertemplate{blocks}[shadow=true]
\setbeamercolor*{block title}{bg=azulloyolaclaro,fg=white}
\setbeamercolor*{block body}{bg=grisclaro,fg=gris}

\setbeamercolor*{block title example}{bg=examplefrente,fg=white}
\setbeamercolor*{block body example}{bg=examplefondo,fg=gris}

\setbeamercolor*{block title alerted}{bg=alertfrente,fg=white}
\setbeamercolor*{block body alerted}{bg=alertfondo,fg=gris}

\usecolortheme[named=azulloyola]{structure}


% This command makes that acrobat reader doesn't changes the colors of the slide
% when there are figures with transparencies.
\pdfpageattr {/Group << /S /Transparency /I true /CS /DeviceRGB>>}
%%%%%%%%%%%%%%%%%%%%%%%%%%%%%%%%%%%%%%%%%%%%%%%%%%%%%%%%%%%%%%%%%%%%%%%%%%%%%%%%

%      + Short title.               + Title which appears in the cover.
%      v                            v
\title[Short title]{Diversidad explícita en modelos de Ensembles de Extreme Learning Machine}
%       + Short author names which appear in the slides.
%       v
\author[Author]
{   % Author names which appear in the cover page.
    Carlos Perales-Gonz\'alez\inst{1}
}
%          + Short affiliation which appears in the slides.
%          v
\institute[ULOYOLA]
{   % Affiliation information which appears in the cover page.
    \begin{tabular}{c}
    \inst{1}Universidad Loyola Andaluc\'ia
    \end{tabular}
}
%     + Short acronym of the conference or date of the presentation.
%     v
\date
{   % Conference name which appears in the cover page.
	\today
}

\begin{document}
% Creates the cover page.
\frame{\titlepage}

\begin{frame}{Esquema}
\tableofcontents
\end{frame}


% 45 minutos
\section{Introducción} % 10 minutos
\subsection{Aprendizaje automático (Machine Learning)}
% 1
\frame{
	
	\frametitle{Inteligencia artificial. Machine Learning}
	
	Qué es la inteligencia artificial. \\
	
	Cómo se enmarca el Machine Learning en la Inteligencia Artificial
	
}
\subsection{Categorías del Aprendizaje Automático}
\frame{
	
	\frametitle{Machine Learning}
	
	\begin{itemize}
		\item Supervisado
		\begin{itemize}
			\item Regression
			\item Classificación
		\end{itemize}
		\item No supervisado
		\begin{itemize}
			\item Asociación
			\item Agrupación
			\item Self-organizing maps
		\end{itemize}
		\item Por refuerzo
	\end{itemize}

En esta tesis, ML supervisado para regresión y clasificación.
	
}

\frame{
\frametitle{Supervised Machine Learning}

Conjunto de datos de entrenamiento $\mathcal{D} = \{  (\boldsymbol{x}_1, y_1), \ldots (\boldsymbol{x}_N, y_N) \} = \{ (\boldsymbol{x}_n, y_n) \}_{n=1}^N$. Buscamos un predictor $f : \mathcal{X} \rightarrow \mathcal{Y}$,

\begin{equation}
\label{eq:f}
f (\boldsymbol{x}) \approx y .
\end{equation}

Hay muchos predictores posibles. Además, dependiendo del problema (regresión o clasificación), esa variable $y$ puede ser una variable numérica, categórica o un vector. 
}
\frame{
Es decir, no podemos conocer exactamente la relación entre el dominio de las features, $\mathcal{X}$, y el dominio target, $\mathcal{Y}$. Sin embargo, podemos interpolar esos dominios a través del conocimiento de algunas de esas relaciones entre features, $\boldsymbol{x} \in \mathcal{X}$, y variable a predecir, $\boldsymbol{y} \in \mathcal{Y}$.

Para entrenar un predictor, minimizamos el error cometido en predecir una serie de valores, que conocemos del set de entrenamiento $\mathcal{D}$.

\begin{equation}
\label{eq:omega_error}
\min_{\theta \in \Theta} \text{Error} ( \left( f(\boldsymbol{x}_1; \theta ), t_1 \right), \ldots, \left( f(\boldsymbol{x}_N; \theta ), t_N \right) )
\end{equation}
}


\subsection{Supervised Machine Learning}

\section{Ensemble learning} % 5 minutos

\section{Extreme Learning Machine}  % 5 minutos


\section{Promoción de la diversidad explícita en ELM}  % 20 minutos
\subsection{RE-ELM}
\subsection{NC-ELM}
\subsection{GNC-ELM}

\section{Conclusiones} % 5 minutos



% END
\section*{END}
\frame{
	\frametitle{END}
\begin{center}
	\huge GRACIAS
\end{center}
}

% BIBLIOGRAPHY
\section*{Bibliography}
\begin{frame}[allowframebreaks]
	\frametitle{References}
    \bibliographystyle{ieeetr}
	\bibliography{workshop_bib.bib}
\end{frame}


\end{document}
